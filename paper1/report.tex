\documentclass[sigconf]{acmart}

\usepackage{hyperref}

\usepackage{endfloat}
\renewcommand{\efloatseparator}{\mbox{}} % no new page between figures

\usepackage{booktabs} % For formal tables

\settopmatter{printacmref=false} % Removes citation information below abstract
\renewcommand\footnotetextcopyrightpermission[1]{} % removes footnote with conference information in first column
\pagestyle{plain} % removes running headers

\begin{document}
\title{Big Data Dangers: Weaponizing Social Media}



\author{Ross Wood}
\affiliation{%
  \institution{rmw@indiana.edu}
  \streetaddress{HID 345}
}

% The default list of authors is too long for headers}
\renewcommand{\shortauthors}{B. Trovato et al.}


\begin{abstract}
Social media has changed the way people get information, disseminate information, communicate, and stay in touch with others both online and in the real world. As more and more people from different age and socioeconomic demographics begin adopting social media and becoming active users, data is being created at a geometric rate. The analysis of all this data being generated can be used for many different purposes, including nefarious ones. It is possible to analyze the digital footprint of social media users in order to accurately target enormous swaths of individuals with propaganda, misinformation, and deception which have been created to cater to their social or political bias.
\end{abstract}

\keywords{Social Media, Social Media Mining, i523, Big Data, Social Media Scraping}


\maketitle

\section{Introduction}

The rise of social media among all tiers of Western society, not just the tech savvy portion, has created interesting opportunities in the field of big data analytics. Indeed, a report found that the population of adults in the United States who use social media rose from 7\% in 2005, to 65\% in 2015 [1]. Indeed, the report found that ``there continues to be growth in social media usages among some groups that were not among the earliest adopters, including older Americans'' [1]. This increase in users creates a tremendous amount of data, all of which can be analyzed to reveal information about the users. This information can be used to inform, educate, and improve the daily systems we use. However, in the wrong hands, this kind of information could be used to influence and manipulate citizens into supporting things that are against their self-interest, and against the interests of their society. 


\section{Discussion}

The increase in size of social media user bases is helping lead the way in 21st century social engineering, for better or worse. The ability to scrape massive amounts of user data from social media sites allows for an analysis of a user's individual digital footprint. When all these footprints are put together and analyzed, conclusions about an individuals' taste in entertainment, political, and social leanings can be drawn, as well as information about an individuals personality. When these conclusions are combined with user location and network structure data, a situation is created where one could manipulate an entire segment of a country's population. This information could then be used to target individuals with misinformation and propaganda that appeals to their confirmation bias, that could then be shared with their like minded friends, creating an self-sustaining cycle of misinformation. The success of this method depends largely on how accurate all the accumulated user data is. The more accurate the data, the better job a machine does at making these predictions. But just how accurate can a machine be at predicting a human's personality and nature based on digital information alone?


\subsection{Accuracy}

The accuracy of a machine's prediction about individual personality and demographics is improved as it accumulates more user data to work with. In essence, the more a person uses social media and creates information about themselves, the easier it is going to be for a machine to look at this information and predict certain things about the person. One study found that to a certain point, humans are better than machines at making personality judgments on other humans. However, once a machine has as little as 100 Facebook likes, its ability to make personality judgments starts to outperform the predictive ability of humans [2]. The study found that with even a small amount of data, machines can predict a person's personality better than that same person's close acquaintance. The study's findings also ``highlight that people's personalities can be predicted automatically and without involving human social-cognitive skills'' [2].
An individual's personality traits are not the only information that machines can gleam from peoples' social media footprints. 

\begin{acks}

  The author

\end{acks}

\bibliographystyle{ACM-Reference-Format}
\bibliography{report} \cite{abril07}

\end{document}

1 - Perrin
2 - Youyou
3 - Preotiuc-Pietro