\documentclass[sigconf]{acmart}

\usepackage{graphicx}
\usepackage{hyperref}
\usepackage{todonotes}

\usepackage{endfloat}
\renewcommand{\efloatseparator}{\mbox{}} % no new page between figures

\usepackage{booktabs} % For formal tables

\settopmatter{printacmref=false} % Removes citation information below abstract
\renewcommand\footnotetextcopyrightpermission[1]{} % removes footnote with conference information in first column
\pagestyle{plain} % removes running headers

\newcommand{\TODO}[1]{\todo[inline]{#1}}

\begin{document}
\title{How Datafication of Activity is Improving Human Health}


\author{Ross Wood}
\affiliation{HID345}
\email{rmw@indiana.edu}

% The default list of authors is too long for headers}
\renewcommand{\shortauthors}{G. v. Laszewski}


\begin{abstract}
As the world becomes more technologically advanced, more and more work is being done on computers digitally versus in the real world physically. This shift in how work is done is creating a situation where huge swaths of people are sitting down for longer periods of time than is medically recommended for healthy living. For many, this has created the need for more exercise and athleticism than what one typically gets in a normal day. This occupational atrophy is one of the driving forces in the development of technology that monitors and datafies the user's day-to-day physical activity and vital signs. All this new data being created is helping make improvements in the lives of those with sedentary jobs, while also revealing new techniques and applications for already existing training or exercise programs. The data is being generated at an increasing rate from the growing number of users who track their activity with the use of sports trackers, smart clothing or athletic wear, and mobile phones. The applications of this data are not limited to improving the health of office workers and other employees whose jobs or lives require low physical activity. The datafication of physical fitness and activity is also changing the way professional athletes and coaches approach their training regiments, as well as having military applications in regards to training personnel.
\end{abstract}

\keywords{i523, HID345, Fitness Tracker, Data Science Exercise, Big Data Exercise, Smart Clothing Data, Activity Data, Smart Clothing }


\maketitle



\section{Introduction}

The big changes taking place as work and labor evolve with technology in the modern world are part of the reason why sports trackers, smart clothes, and mobile health apps are gaining in popularity. For someone who has to work at a computer for six to eight hours a day, a digital reminder to be active, the ability to monitor heart rate, caloric intake/output, or even just steps, can create a huge difference for the individual's health. For people who find themselves in jobs that are not physically demanding and who want to build their athleticism and improve their health, sports tracking and the data they generate can be a good way to stay on track. Through the analysis of personal data generated through activity, users can track their progress, discover and improve areas of difficulty, and reach their exercise goals more easily and efficiently. The technology being created to achieve this, and subsequent data that is generated from it's use, is good for more than just helping people who want to get into shape achieve there goals. All this new technology and the benefits that come with analyzing the data it produces is beginning to be seen in every faucet of society. It is affecting everything from the world of professional sports, to nursing homes, to having some military applications. A new age of personal health is being ushered into reality thanks largely to the ability to create, track, monitor, and analyze all this new activity data.

\section{Usefulness of Sports Trackers}

Sports trackers are tools that users wear to monitor their activities. Colloquially referred to as a Fitbit, these devices are anything that is worn externally that creates data from tracking activity. More and more mobile phones are now coming with standard hardware that enables basic activity tracking and users are taking advantage of these updates by downloading and using apps to track their activity as they progress towards their exercise goal. Smart clothing is a growing field that is presently in its infancy. Smart clothes are any clothes that have sensors built into them that monitor and creates data on human activity and can monitor vital signs. These blooming technologies are paving the way towards a new understanding of personal health.

Studies have shown that activity tracking technology has proven to increase activity in adults who are overweight or would be classified as obese \cite{devries2016}. By reinforcing healthy habits and activities, the sports trackers, apps, or clothes help to encourage a routine of healthy living. Other features, like social media networks associated with tracking apps, have also proven to have a positive influence on user activity \cite{ojala2010}. With easy access to social sports knowledge and these activity tracking tools, humans are entering into a personal health era the likes of which have never been seen before.

\section{Home Consumer Health Benefits}

The average consumer working a sedentary job is one of the groups benefiting most from the adoption of technologies that track and record all the activity data one creates throughout the day. Be it a wrist sport tracker, mobile phone, or a new article of smart clothing, an individual having access to the activity data they generate is important because they can use it to monitor their health, improve their lives, and achieve their fitness goals. With access to their own data, the home consumer gets to become a junior data scientists as they analyze their activity data to improve their performance. The constant ability to assess and monitor their progress and performance is allowing users to shorten plateau time and reach their goals even faster.

The amount of data being generated by active users is growing at an exponential rate as more consumers begin adopting these activity tracking devices. In December of 2015, it was estimated that the amount of new data being generated every day from athletic wear, phones, or trackers ``can easily reach billions of tuples per day'' \cite{cortes2014}. This number is going to continue to rise as the technology improves. The idea of athletic wear via smart clothing is still more or less in in its infancy, but once it starts to become more popular and cost efficient, the amount of data generated is going to start increasing at an even faster rate.

Indeed, the amount of new data being created every day is showing no signs of slowing down. The smart clothing market, which has been steadily growing since 2015, is expected to overtake mobile phones and sports trackers in generating activity data \cite{hanuska2015}. Since ``clothes outsell phones''\cite{hanuska2015}, this data explosion and all the benefits that come with it are only going to continue to boom in the coming decades.

\subsection{Senior Benefits}

These technologies are doing more than just benefiting those who want to achieve greater athleticism or get into shape. They are also helping seniors approach individual personal health in their twilight years in ways no generation has been able to in the past. With access to trackers that monitor their fitness levels and up to the minute details like heart rate, steps, or calories burned, seniors are better able to monitor their own health. This ability to monitor their health has a cascading effect in that better health and fitness has other benefits like stronger bones and better coordination, which can prevent falls and other hazards \cite{valenzuela2016}. As medicine and advances in health care continue to improve and the human lifespan gets longer and longer, these technologies are proving to be a major benefit to groups whose health is at the greatest risk.

Right now, a reticence to adopt new technology is one of the biggest setbacks preventing seniors from taking advantage of activity tracking and the benefits that come with it. This is primarily caused by how quickly technology changes and difficulties learning a new technology. As time goes by and younger generations who have grown up with hand held devices and sports trackers become older, this setback will become less of a problem \cite{valenzuela2016}. As individuals exercise greater control over their own health with help from these tools, humans are going to continue to prolong their lifespans, further shortening the divide between number of years and number of healthy, active years.

\subsection{Social Media User Benefits}

Another group that benefits from the use of this technology is social media users. Online activity and athletic social media communities are one of the fastest growing online social media communities \cite{ojala2010}. Research has found that creating a social network around sports tracking is a novel approach to motivating users to engage in activity more, worry more about their diets and physical condition, while also producing, accessing, and learning important sports knowledge from the online sports community \cite{ojala2010}. Since the primary way to interact with online friends on a social network built around tracking your activity is to be active, this again contributes to the cascading effect in that people will want to exercise more to interact with their online friends more. This is motivating people to push themselves harder and achieve greater levels of physical health.

\subsection{Athletics Organization Benefits}

Health and fitness are not the only areas where this newly created activity data is helpful. Sports organizations from the high school level to professionals are improving their teams' performances from analysis of this data. The technology to track an individual athlete's performance is getting more advanced. ``Sensor technology in sports equipment such as basketballs or golf clubs also allows us to get feedback on our game . . . using smart technology to track nutrition and sleep, as well as social media conversation to monitor emotional well being'' \cite{marr2015}. This new approach to monitoring athletes and their activity is more complex, drills deeper, and is proving to be vastly superior to previous methods of statistical analysis used to improve performance \cite{marr2015}. Starting in the early 2000s, teams that have begun taking advantage of these techniques have all improved their performances. 

\subsection{Military Benefits}

Using many of the same techniques that athletic organizations are using, militaries around the world are beginning to adopt a lot of the same approaches when training their soldiers. As technology that works in sync with human activity becomes more advanced, so too does the modern soldier's reliance upon it in order to maintain a combative edge. Smart clothing that can monitor for potentially hazardous environmental conditions, as well as physical and mental health and safety are among some of the technologies being developed that assist the user and create data from monitoring human activity \cite{scataglini2015}. The United States military is dedicated to the development of these smart clothing technologies that will benefit their soldiers \cite{lester2011}. An example would be an article of clothing that can detect an injury and automatically call for help, or even administer some rudimentary medical treatment to help save the solder's life. As is often the case, the development of these technologies frequently find their way to the private sector and civilian use. Thus, even more data and technology to monitor human activity is on the horizon.

\section{Privacy Concerns}

As is usually the case with all the data creation in the 21st century, it has the potential to be a double edged sword. While undoubtedly useful to the individual user and society as a whole, the data created from sports trackers, smart clothes, and mobile phones are also readily available to the technology and software creators \cite{lidynia2017}. An example of how this data could be used against the user is through sharing the user data with other organizations. One example of how this data could hurt the user would be if an individual's personal heath or activity information were somehow made available to another companies who could potentially gouge their customers based on unhealthy activity or any other number of variables. Even though the data is private, it is possible to use machine learning techniques to identify users around the world. There are few modern laws that govern what companies and organizations are allowed to do with their user generated data. This type of privacy concern is just now beginning to be addressed in courts around the world, but for now, it remains a valid point of concern for the ever growing number of users. 

\section{Conclusion}

Fitness tracking technology is making it easier for humans to achieve their fitness goals in ways they have never been able to in the past. With new research and development being done everyday to improve the already existing tools and techniques, the technologies are only going to continue to get more precise and efficient in regards to helping humans monitor their activity and quantify their vital signs. The myriad of organizations discovering the usefulness of this approach to physical data analytics is pushing this field to greater heights every year. Monitoring and studying our individual activity data is shaping up to be the key to having a long and healthy life.

\begin{acks}

  The author would like to thank Dr. Gregor von Laszewski, Miao Jiang, and Juliette Zerick for assistance with this assignment and using github.

\end{acks}


\bibliographystyle{ACM-Reference-Format}
\bibliography{report} 



\section{Issues}

\DONE{Example of done item: Once you fix an item, change TODO to DONE}

\subsection{Assignment Submission Issues}

    \DONE{Do not make changes to your paper during grading, when your repository should be frozen.}

\subsection{Uncaught Bibliography Errors}

    \DONE{Missing bibliography file generated by JabRef}
    \DONE{Bibtex labels cannot have any spaces, \_ or \& in it}
    \DONE{Citations in text showing as [?]: this means either your report.bib is not up-to-date or there is a spelling error in the label of the item you want to cite, either in report.bib or in report.tex}

\subsection{Formatting}

    \DONE{Incorrect number of keywords or HID and i523 not included in the keywords}
    \DONE{Other formatting issues}

\subsection{Writing Errors}

    \DONE{Errors in title, e.g. capitalization}
    \DONE{Spelling errors}
    \DONE{Are you using {\em a} and {\em the} properly?}
    \DONE{Do not use phrases such as {\em shown in the Figure below}. Instead, use {\em as shown in Figure 3}, when referring to the 3rd figure}
    \DONE{Do not use the word {\em I} instead use {\em we} even if you are the sole author}
    \DONE{Do not use the phrase {\em In this paper/report we show} instead use {\em We show}. It is not important if this is a paper or a report and does not need to be mentioned}
    \DONE{If you want to say {\em and} do not use {\em \&} but use the word {\em and}}
    \DONE{Use a space after . , : }
    \DONE{When using a section command, the section title is not written in all-caps as format does this for you}\begin{verbatim}\section{Introduction} and NOT \section{INTRODUCTION} \end{verbatim}

\subsection{Citation Issues and Plagiarism}

    \DONE{It is your responsibility to make sure no plagiarism occurs. The instructions and resources were given in the class}
    \DONE{Claims made without citations provided}
    \DONE{Need to paraphrase long quotations (whole sentences or longer)}
    \DONE{Need to quote directly cited material}

\subsection{Character Errors}

    \DONE{Erroneous use of quotation marks, i.e. use ``quotes'' , instead of " "}
    \DONE{To emphasize a word, use {\em emphasize} and not ``quote''}
    \DONE{When using the characters \& \# \% \_  put a backslash before them so that they show up correctly}
    \DONE{Pasting and copying from the Web often results in non-ASCII characters to be used in your text, please remove them and replace accordingly. This is the case for quotes, dashes and all the other special characters.}
    \DONE{If you see a figure and not a figure in text you copied from a text that has the fi combined as a single character}

\subsection{Structural Issues}

    \DONE{Acknowledgement section missing}
    \DONE{Incorrect README file}
    \DONE{In case of a class and if you do a multi-author paper, you need to add an appendix describing who did what in the paper}
    \DONE{The paper has less than 2 pages of text, i.e. excluding images, tables and figures}
    \DONE{The paper has more than 6 pages of text, i.e. excluding images, tables and figures}
    \DONE{Do not artificially inflate your paper if you are below the page limit}

\subsection{Details about the Figures and Tables}

    \DONE{Capitalization errors in referring to captions, e.g. Figure 1, Table 2}
    \DONE{Do use {\em label} and {\em ref} to automatically create figure numbers}
    \DONE{Wrong placement of figure caption. They should be on the bottom of the figure}
    \DONE{Wrong placement of table caption. They should be on the top of the table}
    \DONE{Images submitted incorrectly. They should be in native format, e.g. .graffle, .pptx, .png, .jpg}
    \DONE{Do not submit eps images. Instead, convert them to PDF}

    \DONE{The image files must be in a single directory named "images"}
    \DONE{In case there is a powerpoint in the submission, the image must be exported as PDF}
    \DONE{Make the figures large enough so we can read the details. If needed make the figure over two columns}
    \DONE{Do not worry about the figure placement if they are at a different location than you think. Figures are allowed to float. For this class, you should place all figures at the end of the report.}
    \DONE{In case you copied a figure from another paper you need to ask for copyright permission. In case of a class paper, you must include a reference to the original in the caption}
    \DONE{Remove any figure that is not referred to explicitly in the text (As shown in Figure ..)}
    \DONE{Do not use textwidth as a parameter for includegraphics}
    \DONE{Figures should be reasonably sized and often you just need to
  add columnwidth} e.g. \begin{verbatim}/includegraphics[width=\columnwidth]{images/myimage.pdf}\end{verbatim}

re

\end{document}
